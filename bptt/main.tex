\documentclass{article}
\usepackage[utf8]{inputenc}
\usepackage{amsmath}


\title{Back Propagation Through Time For Vanilla Recurrent Neural Network}
\author{Yongqiang Huang}
\date{December 2, 2019}

\begin{document}

\maketitle
\section{Scalar form}
First we look at the over-simplified scalar version where all variables are one-dimensional. Let our network be a binary classifier:
\begin{align}
    s_t &= tanh(Ux_t + Ws_{t-1}) \label{eq-recurrent} \\
    \hat{y}_t &= sigm(Vs_t) \\
    E_t &= -y_t\log \hat{y}_t
\end{align}
Assume $t \geq 1$ and $s_0$ is given, that is $s_0$ does not depend on any other variable. We want to compute $\frac{\partial E_t}{\partial V}$, $\frac{\partial E_t}{\partial W}$, and  $\frac{\partial E_t}{\partial U}$. First, let us compute $\frac{\partial E_t}{\partial V}$. 
\begin{equation}
    \frac{\partial E_t}{\partial V} = \frac{\partial E_t}{\partial \hat{y}_t}\frac{\partial \hat{y}_t}{\partial V}
\end{equation}
Then, we compute $\frac{\partial E_t}{\partial W}$:
\begin{equation}
    \frac{\partial E_t}{\partial W} = \frac{\partial E_t}{\partial \hat{y}_t}\frac{\partial \hat{y}_t}{\partial s_t}\frac{\partial s_t}{\partial W}
\end{equation}
Notice from Eq. \eqref{eq-recurrent} that $s_t$ is a function of both $W$ and $s_{t-1}$. Before going forward, we need to review the concept of total derivative, which says the following. If function $z = f(u, v)$ where $u = u(x)$, $v = v(x)$, then the total derivative of $z$ with respect to $x$ is
\begin{equation} \label{eq-total_derivative}
    \frac{dz}{dx} = \frac{\partial z}{\partial u}\frac{du}{dx} + \frac{\partial z}{\partial v}\frac{dv}{dx}
\end{equation}
Notice the definition does not confine the exact form of $f(\cdot)$, it does not have to be addition, nor does it have to be multiplication, or specifically anything else. As long as it involves different functions of $x$, the functions would contribute to the total derivative linearly with the same weight 1. Okay, back to what we were saying. $s_t = tanh(Ux_t + Ws_{t-1})$, here $W$ is a function of $W$ itself, and $s_{t-1}$ is a function of $W$, so using the definition of the total derivative in Eq. \eqref{eq-total_derivative}, we have 
\begin{equation}
    \frac{\partial s_t}{\partial W} = \frac{\partial s_t}{\partial W}\frac{\partial W}{\partial W}  + \frac{\partial s_t}{\partial s_{t-1}}\frac{\partial s_{t-1}}{\partial W}   
\end{equation}
Hence the derivative is recursive, and it stops at $s_1$ which depends on $s_0$. For example, $\frac{\partial s_3}{\partial W}$ expands to 
\begin{align}
    \frac{\partial s_3}{\partial W} &= \frac{\partial s_3}{\partial W} + \frac{\partial s_3}{\partial s_2}\frac{\partial s_2}{\partial W} + \frac{\partial s_3}{\partial s_2}\frac{\partial s_2}{\partial s_1}\frac{\partial s_1}{\partial W}\\
    &= \frac{\partial s_3}{\partial s_3}\frac{\partial s_3}{\partial W} + \frac{\partial s_3}{\partial s_2}\frac{\partial s_2}{\partial W} + \frac{\partial s_3}{\partial s_1}\frac{\partial s_1}{\partial W} \\
    &= \sum_{1\leq k \leq 3}\frac{\partial s_3}{\partial s_k}\frac{\partial s_k}{\partial W}
\end{align}
The expansion shows that the $\frac{\partial s_3}{\partial W}$ goes through $s_3$, $s_2$, and $s_1$, in every $s$ for which $W$ is an input. We can generalize the formula of $s_3$ to $s_t$:
\begin{align}
    \frac{\partial s_t}{\partial W} &= \sum_{1\leq k \leq t}\frac{\partial s_t}{\partial s_k}\frac{\partial s_k}{\partial W} \label{eq-dw_sum1}\\
    &= \frac{\partial s_t}{\partial s_t}\frac{\partial s_t}{\partial W} + \frac{\partial s_t}{\partial s_{t-1}}\frac{\partial s_{t-1}}{\partial W} + \frac{\partial s_t}{\partial s_{t-2}}\frac{\partial s_{t-2}}{\partial W} + \dots +  \frac{\partial s_t}{\partial s_1}\frac{\partial s_1}{\partial W}
\end{align}
Let's take a closer look at each individual term $\frac{\partial s_t}{\partial s_k}\frac{\partial s_k}{\partial W}$. If we expand it, we have
\begin{align}
    \frac{\partial s_t}{\partial s_k}\frac{\partial s_k}{\partial W} &= \frac{\partial s_t}{\partial s_{t-1}}\frac{\partial s_{t-1}}{\partial s_{t-2}}\frac{\partial s_{t-2}}{\partial s_{t-3}}\cdots\frac{\partial s_{k+1}}{\partial s_k}\frac{\partial s_k}{\partial W} \\
    &= W\cdot W\cdot W\cdots \frac{\partial s_k}{\partial W} \\
    &= W^{t - k}\frac{\partial s_k}{\partial W}
\end{align}
Thus we can rewrite Eq. \eqref{eq-dw_sum1} as 
\begin{equation}
    \frac{\partial s_t}{\partial W} = \sum_{1\leq k \leq t}W^{t - k}\frac{\partial s_k}{\partial W}    
\end{equation}
Lastly, we compute $\frac{\partial E_t}{\partial U}$. $U$ is an input to $s_t$, which means we will treat $U$ similarly to the way we treat $W$:
\begin{equation}
    \frac{\partial E_t}{\partial U} = \frac{\partial E_t}{\partial \hat{y}_t}\frac{\partial \hat{y}_t}{\partial s_t}\frac{\partial s_t}{\partial U}
\end{equation}
where 
\begin{equation}
    \frac{\partial s_t}{\partial U} = \sum_{1\leq k \leq t}W^{t - k}\frac{\partial s_k}{\partial U}    
\end{equation}

\section{Vector form, single data point}
Now that we are comfortable with the scalar version, we are proceeding to the vector version, which requires us to define the network again, a little differently:
\begin{align}
    z_t &= Ux_t + Ws_{t-1} \\
    s_t &= \text{tanh}(z_t) \\
    g_t &= Vs_t \\
    \hat{y}_t &= \text{softmax}(g_t) \\
    L_t &= -y^\top_t\log(\hat{y}_t)
\end{align}
The dimensions of the variables are
\begin{align}
    x_t&: M\times 1\\ 
    z_t, s_t&: D\times 1\\
    g_t, \hat{y}_t, y_t&: C\times 1\\
    U&: D\times M\\
    W&:  D\times D\\
    V&: C\times D
\end{align}
 $L_t$ is the loss at time $t$, which can also be written using summation:
\begin{equation}
    L_t = -\sum_{i=1}^{C}y_{t, i}\log(\hat{y}_{t, i})
\end{equation}
The back propagation starts with  $\frac{\partial L_t}{\partial \hat{y}_t}$, which has dimension $1\times C$: we have only one output variable $L_t$ and $C$ input variables $\hat{y}_{t, 1}, \hat{y}_{t, 2}, \cdots, \hat{y}_{t, C}$.
\begin{equation}
    \frac{\partial L_t}{\partial \hat{y}_t} = [\frac{\partial L_{t, 1}}{\partial \hat{y}_{t, 1}}, \frac{\partial L_{t, 2}}{\partial \hat{y}_{t, 2}}, \cdots, \frac{\partial L_{t, C}}{\partial \hat{y}_{t, C}}]
\end{equation}
If we consider $\log() = \text{ln}()$, i.e. natual logarithm, then 
\begin{equation}
    \frac{\partial L_t}{\partial \hat{y}_t} = [-\frac{y_{t, 1}}{\hat{y}_{t, 1}}, -\frac{y_{t, 2}}{\hat{y}_{t, 2}}, \cdots, -\frac{y_{t, C}}{\hat{y}_{t, C}}]
\end{equation}
Next, we compute $\frac{\partial L_t}{\partial g_t}$:
\begin{equation}
    \frac{\partial L_t}{\partial g_t} = \frac{\partial L_t}{\partial \hat{y}_t}\frac{\partial \hat{y}_t}{\partial g_t}
\end{equation}
Let's get the dimension right. $\frac{\partial L_t}{\partial g_t}$ should also have dimension $1\times C$, and $\frac{\partial \hat{y}_t}{\partial g_t}$ should have dimension $C\times C$. The relation between $\hat{y}_t$ and $g_t$ is as follows:
\begin{equation}
    \hat{y}_{t, i} = \frac{\exp(g_{t, i})}{\sum_{i=j}^C\exp(g_{t, j})}
\end{equation}
and the derivative
\begin{equation}
    \frac{\partial \hat{y}_{t, i}}{\partial g_{t, j}} = 
    \begin{cases}
    \hat{y}_{t, i}(1 - \hat{y}_{t, i}) & \text{if }i = j\\
    -\hat{y}_{t, i}\hat{y}_{t, j} & \text{if }i \neq j
    \end{cases}
\end{equation}
or 
\begin{equation}
    \frac{\partial \hat{y}_{t, i}}{\partial g_{t, j}} = \hat{y}_{t, i}(\mathbf{1}(i=j) - \hat{y}_{t, j})    
\end{equation}
where 
\begin{equation}
    \mathbf{1}(i=j) = 
    \begin{cases}
    1 & \text{if }i = j \\
    0 & \text{if }i\neq j
    \end{cases}
\end{equation}
$\frac{\partial \hat{y}_{t, i}}{\partial g_{t, j}}$ is an element representation of matrix $\frac{\partial \hat{y}_t}{\partial g_t}$, it shows everything we need to know about matrix $\frac{\partial \hat{y}_t}{\partial g_t}$. Now let's compute $\frac{\partial L_t}{\partial g_t}$. To do that we only need to know its elements:
\begin{equation}
    \frac{\partial L_t}{\partial g_{t, i}} = \sum_{k=1}^C\frac{\partial L_t}{\partial \hat{y}_{t, k}}\frac{\partial \hat{y}_{t, k}}{\partial g_{t, i}}
\end{equation}
Let's pause here and see what it means. $L_t$ goes to $g_{t_{t, i}}$ through $\hat{y}_t$, and more specifically, through \emph{every} element of $\hat{y}_t$. Therefore, we need to consider  \emph{every} element of $\hat{y}_t$. Let's continue.
\begin{align}
     \frac{\partial L_t}{\partial g_{t, i}} &=
     -\sum_{k=1}^C\frac{y_{t, k}}{\hat{y}_{t, k}}\hat{y}_{t, k}(\mathbf{1}(k=i) - \hat{y}_{t, i}) \\
     &= -\sum_{k=1}^Cy_{t, k}(\mathbf{1}(k=i) - \hat{y}_{t, i}) \\
     &= \sum_{k=1}^Cy_{t, k}\hat{y}_{t, i} - y_{t, i}\\
     &= (\sum_{k=1}^Cy_{t, k})\hat{y}_{t, i} - y_{t, i} \label{eq-1} \\
     \frac{\partial L_t}{\partial g_{t, i}} &= \hat{y}_{t, i} - y_{t, i} \label{eq-2}
\end{align}
From Eq. \eqref{eq-1} to Eq. \eqref{eq-2}, we assume $y_t$ is a one-hot vector and therefore $\sum_{k=1}^Cy_{t, k} = 1$. In vector form:
\begin{equation}
\frac{\partial L_t}{\partial g_t} = (\hat{y}_t - y_t)^\top    
\end{equation}
Now we are ready to compute the gradient of our first weight matrix $V$:
\begin{equation}
    \frac{\partial L_t}{\partial V} = \frac{\partial L_t}{\partial g_t} \frac{\partial g_t}{\partial V}
\end{equation}
Now there is a problem: what is $\frac{\partial g_t}{\partial V}$? Apparently its dimension should be $C\times (C\times D)$, which is intimidating, and that is why we will \emph{not} directly compute it. After all, we only care about $\frac{\partial L_t}{\partial V}$, which only requires that we know its element $\frac{\partial L_t}{\partial V_{i, j}}$. This is so important that it deserves our pausing here to strengthen it:

We \emph{never} specifically write out a tensor, which has more than two dimensions. Rather, we represent the tensor using only its elements, which is equivalent. Let the input be $x$, and the output be $y$, where
\begin{equation}
    y = Wx
\end{equation}
and let the final loss be $L$, which is a scalar. We strive to always compute $\frac{\partial L}{\partial y}$ first, which for now, we assume to be a vector. Then we compute $\frac{\partial L}{\partial W}$ by computing $\frac{\partial L}{\partial W_{i, j}}$. We always keep a scalar(loss)-to-variable at hand before we go further in the back propagation. By the way, this makes the back-propagating process sequential. 


Back to the derivation, $L_t$ goes to $V_{i, j}$ through every single element in $g_t$, of which the consequence we sum up:
\begin{align}
    \frac{\partial L_t}{\partial V_{i, j}} &= \sum_{k=1}^C \frac{\partial L_t}{\partial g_{t, k}} \frac{\partial g_{t, k}}{\partial V_{i, j}}
\end{align}
To know $\frac{\partial g_{t, k}}{\partial V_{i, j}}$, we need to know how $g_t$ is computed from $V$. A row in $g_t$ only uses the same row in $V$:
\begin{equation}
    g_{t, i} = \sum_{d = 1}^DV_{i, d}s_{t, d}, \quad (i=1, 2, \dots, C)
\end{equation}
Therefore if we look at one row in $g_t$ and a different row in $V$, the derivative will be 0. To summarize:
\begin{align}
    \frac{\partial g_{t, k}}{\partial V_{i, j}} &= 
    \begin{cases}
    s_{t, j} & \text{if } i = k \\
    0 & \text{if } i\neq k
    \end{cases} \\
    &= \mathbf{1}(i=k)s_{t, j}
\end{align}
Now let's get back to solving $\frac{\partial L_t}{\partial V_{i, j}}$. 
\begin{align}
    \frac{\partial L_t}{\partial V_{i, j}} &= \sum_{k=1}^C \frac{\partial L_t}{\partial g_{t, k}} \frac{\partial g_{t, k}}{\partial V_{i, j}} \\
    &= \sum_{k=1}^C (\hat{y}_{t, k} - y_{t, k})\mathbf{1}(i=k)s_{t, j} \\
    &= (\hat{y}_{t, i} - y_{t, i})s_{t, j}
\end{align}
Thus, we can obtain the vector form of the gradient, which is an outer product:
\begin{equation}
    \frac{\partial L_t}{\partial V} = (\hat{y}_t - y_t)s_t^\top
\end{equation}
Next we compute $\frac{\partial L_t}{\partial W}$, which is 
\begin{equation}
    \frac{\partial L_t}{\partial W} = \frac{\partial L_t}{\partial s_t}\frac{\partial s_t}{\partial W}
\end{equation}
$\frac{\partial L_t}{\partial s_t}$ is computed by
\begin{align}
    \frac{\partial L_t}{\partial s_t} &= \frac{\partial L_t}{\partial g_t}\frac{\partial g_t}{\partial s_t}
    = (\hat{y}_t - y_t)^\top V
\end{align}
$\frac{\partial s_t}{\partial W}$ involves recursive computation. Different from the scalar case, we will not write out the recursion naively, because it will be multi-dimensional. We will, as previously mentioned, write the recursion as part of the computation of a scalar-to-matrix chain:
\begin{align}
    \frac{\partial L_t}{\partial W} &= \sum_{k=1}^t \frac{\partial L_t}{\partial s_t}\frac{\partial s_t}{\partial s_k}\frac{\partial s_k}{\partial W} 
\end{align}
Let's break the above equation apart and make it more specific. First we look at $\frac{\partial s_t}{\partial s_k}$:
\begin{align}
    \frac{\partial s_t}{\partial s_k} &= \frac{\partial s_t}{\partial s_{t-1}}\frac{\partial s_{t-1}}{\partial s_{t-2}}\cdots\frac{\partial s_{k+1}}{\partial s_k} \\
    &= \prod_{i=k+1}^t\frac{\partial s_i}{\partial s_{i-1}} \\
    &= \prod_{i=k+1}^t\frac{\partial s_i}{\partial z_i}\frac{\partial z_i}{\partial s_{i-1}} \\
\end{align}
Since tanh() is element-wise, $\frac{\partial s_i}{\partial z_i}$ is a diagonal matrix:
\begin{equation}
    \frac{\partial s_i}{\partial z_i} = \text{diag}((1-s_{i, 1}^2), (1-s_{i, 2}^2), \cdots, (1-s_{i, D}^2))
\end{equation}
Thus
\begin{align}
    \frac{\partial s_t}{\partial s_k} &= \prod_{i=k+1}^t\frac{\partial s_i}{\partial z_i}W = W^{t-k}\prod_{i=k+1}^t\frac{\partial s_i}{\partial z_i}
\end{align}
The immediate derivative of $s_k$ with respect to $W$ is
\begin{equation}
    \frac{\partial s_k}{\partial W} = \frac{\partial s_k}{\partial z_k}\frac{\partial z_k}{\partial W}  
\end{equation}
Thus the specific form of $\frac{\partial L_t}{\partial W}$ is
\begin{align}
    \frac{\partial L_t}{\partial W} &= \sum_{k=1}^t \frac{\partial L_t}{\partial s_t}\frac{\partial s_t}{\partial s_k}\frac{\partial s_k}{\partial W}\\
    &= \sum_{k=1}^t\frac{\partial L_t}{\partial s_t}W^{t-k}\left(\prod_{i=k+1}^t\frac{\partial s_i}{\partial z_i}\right)\frac{\partial s_k}{\partial z_k}\frac{\partial z_k}{\partial W}\\
    \frac{\partial L_t}{\partial W} &= \sum_{k=1}^t\frac{\partial L_t}{\partial s_t}W^{t-k}\left(\prod_{i=k}^t\frac{\partial s_i}{\partial z_i}\right)\frac{\partial z_k}{\partial W}
\end{align}
The gradient of $U$ is very similar to that of $W$:
\begin{align}
    \frac{\partial L_t}{\partial U} &= \sum_{k=1}^t \frac{\partial L_t}{\partial s_t}\frac{\partial s_t}{\partial s_k}\frac{\partial s_k}{\partial U} \\
    &= \sum_{k=1}^t\frac{\partial L_t}{\partial s_t}W^{t-k}\left(\prod_{i=k}^t\frac{\partial s_i}{\partial z_i}\right)\frac{\partial z_k}{\partial U}
\end{align}
One detail that will be used in the implementation of the above is, given $y = Wx$ and $\frac{\partial L}{\partial y}$, what is $\frac{\partial L}{\partial W}$? The answer is
\begin{equation}
   \frac{\partial L}{\partial W} = (x\cdot\frac{\partial L}{\partial y})^\top  
\end{equation}
The final gradient with respect to the weights are:
\begin{align}
    \frac{\partial L}{\partial V} &= \sum_{t=1}^T\frac{\partial L_t}{\partial V} \\
    \frac{\partial L}{\partial W} &= \sum_{t=1}^T\frac{\partial L_t}{\partial W} \\
    \frac{\partial L}{\partial U} &= \sum_{t=1}^T\frac{\partial L_t}{\partial U}
\end{align}

\section{Vector form, multiple data points}
For multiple data points, we would like to slightly change the formulation again. Specifically, we use $X$ to represent the data, which has $N$ points, and each has $D$ dimensions. Thus, the dimensions of $X$ is $N * D$. The network is written as follows:
\begin{align}
    Z(t) &= X(t)\cdot U + S(t-1)\cdot W \\
    S(t) &= \tanh(Z(t)) \\
    G(t) &= S(t)\cdot V \\
    \hat{Y}(t) &= \text{softmax}(G(t)) \\
    L(t) &= -\sum_{i=1}^N\sum_{j=1}^CY(t)_{i, j}\log(\hat{Y}_{i, j})
\end{align}
and the dimensions of the variables are:
\begin{align}
    X(t) &: N \times D \\
    Z(t), S(t) &: N \times M \\
    G(t), \hat{Y}(t), Y(t) &: N \times C \\
    U &: D \times M \\
    W &: M \times M \\
    V &: M \times C
\end{align}
We will follow the order of single-data-point vector form. First we compute $\frac{\partial L(t)}{\partial \hat{Y}(t)}$ which has dimension $N \times C$. Assuming $\log() = \text{ln}()$:
\begin{equation}
    \frac{\partial L(t)}{\hat{Y}(t)_{i, j}} = -\frac{Y(t)_{i, j}}{\hat{Y}(t)_{i, j}}
\end{equation}
Then
\begin{equation}
    \frac{\partial \hat{Y}(t)_{m, n}}{\partial G(t)_{i, j}} = \mathbf{1}(m = i)\hat{Y}(t)_{m, n}(\mathbf{1}(n=j) - \hat{Y}(t)_{m, j}) 
\end{equation}
and
\begin{align}
    \frac{\partial L(t)}{\partial G(t)_{i, j}} 
    &= \sum_{m=1}^N\sum_{n=1}^C\frac{\partial L(t)}{\hat{Y}(t)_{m, n}}\frac{\partial \hat{Y}(t)_{m, n}}{\partial G(t)_{i, j}} \\
    &= -\sum_{m=1}^N\sum_{n=1}^C\frac{Y(t)_{m, n}}{\hat{Y}(t)_{m, n}}\mathbf{1}(m = i)\hat{Y}(t)_{m, n}(\mathbf{1}(n=j) - \hat{Y}(t)_{m, j}) \\
    &= -\sum_{m=1}^N\sum_{n=1}^CY(t)_{m, n}\mathbf{1}(m = i)(\mathbf{1}(n=j) - \hat{Y}(t)_{m, j}) \\
    &= -\sum_{n=1}^CY(t)_{i, n}(\mathbf{1}(n=j) - \hat{Y}(t)_{i, j}) \\
    &= \hat{Y}(t)_{i, j} - Y(t)_{i, j}
\end{align}
Thus 
\begin{equation}
    \frac{\partial L(t)}{\partial G(t)} = \hat{Y}(t) - Y(t) 
\end{equation}
The gradient with respect to $V$ is computed by
\begin{equation}
    \frac{\partial L(t)}{\partial V_{i, j}} = \sum_{m=1}^N\sum_{n=1}^C \frac{\partial L(t)}{\partial G_{m, n}} \frac{\partial G_{m, n}}{\partial V_{i, j}}
\end{equation}
We know that 
\begin{equation}
    G(t)_{m, n} = \sum_{k=1}^M S(t)_{m, k}V_{k, n}
\end{equation}
and therefore
\begin{equation}
    \frac{\partial G_{m, n}}{\partial V_{i, j}} = \mathbf{1}(n = j)S(t)_{m, i}
\end{equation}
Thus, the gradient with respect to $V$ is 
\begin{align}
    \frac{\partial L(t)}{\partial V_{i, j}} 
    &= \sum_{m=1}^N\sum_{n=1}^C \frac{\partial L(t)}{\partial G_{m, n}} \frac{\partial G_{m, n}}{\partial V_{i, j}} \\
    &= \sum_{m=1}^N\sum_{n=1}^C (\hat{Y}(t)_{m, n} - Y(t)_{m, n}) \mathbf{1}(n = j)S(t)_{m, i} \\
    &= \sum_{m=1}^N (\hat{Y}(t)_{m, j} - Y(t)_{m, j}) S(t)_{m, i} \\
    &= \sum_{m=1}^N S(t)_{m, i}(\hat{Y}(t)_{m, j} - Y(t)_{m, j}) 
\end{align}
Thus 
\begin{equation}
    \frac{\partial L(t)}{\partial V} = S(t)^\top(\hat{Y}(t) - Y(t))
\end{equation}
The gradient with respect to $W$ is
\begin{align}
    \frac{\partial L(t)}{\partial W} 
    &= \sum_{k=1}^t 
    \frac{\partial L(t)}{\partial S(t)} 
    \frac{\partial S(t)}{\partial S(k)}
    \frac{\partial S(k)}{\partial W} \\
    &= \sum_{k=1}^t 
    \frac{\partial L(t)}{\partial S(t)}
    \frac{\partial S(t)}{\partial Z(t)}
    \frac{\partial Z(t)}{\partial S(k)}
    \frac{\partial Z(k)}{\partial W}
\end{align}
There are a few terms we must first handle. First,
\begin{equation}
    \frac{\partial S(t)_{m, n}}{\partial Z(t)_{i, j}} = \mathbf{1}(m=i, n=j)(1 - S(t)_{i, j}^2)
\end{equation}
Then 
\begin{equation}
    \frac{\partial L(t)}{\partial S(t)} = \frac{\partial L(t)}{\partial G(t)}V^\top = (\hat{Y}(t) - Y(t))V^\top
\end{equation}
and then 
\begin{align}
    \frac{\partial L(t)}{\partial Z(t)_{i, j}} 
    &= \sum_{m=1}^N\sum_{n=1}^M
    \frac{\partial L(t)}{\partial S(t)_{m, n}}
    \frac{\partial S(t)_{m, n}}{\partial Z(t)_{i, j}} \\
    &= \sum_{m=1}^N\sum_{n=1}^M
    \frac{\partial L(t)}{\partial S(t)_{m, n}}
    \mathbf{1}(m=i, n=j)(1 - S(t)_{i, j}^2) \\
    &= \frac{\partial L(t)}{\partial S(t)_{i, j}}
    (1 - S(t)_{i, j}^2)
\end{align}
Thus 
\begin{equation} \label{eq-lToZ}
    \frac{\partial L(t)}{\partial Z(t)} 
    = \frac{\partial L(t)}{\partial S(t)} \circ (I - S(t) \circ S(t))
\end{equation}
where $\circ$ represents element-wise multiplication. Now let's look at $\frac{\partial Z(t)}{\partial S(t-1)}$:
\begin{equation} \label{eq-lToS}
    \frac{\partial L(t)}{\partial S(t-1)} = \frac{\partial L(t)}{\partial Z(t)}\frac{\partial Z(t)}{\partial S(t-1)} = \frac{\partial L(t)}{\partial Z(t)}W^\top
\end{equation}
Going back to the gradient with respect to $W$, we have
\begin{align}
    \frac{\partial L(t)}{\partial W} 
    &= \sum_{k=1}^t 
    \frac{\partial L(t)}{\partial S(t)}
    \frac{\partial S(t)}{\partial Z(t)}
    \frac{\partial Z(t)}{\partial S(k)}
    \frac{\partial Z(k)}{\partial W} \\
    &= \sum_{k=1}^t 
    \frac{\partial L(t)}{\partial S(t)}
    \frac{\partial S(t)}{\partial Z(t)}
    \frac{\partial Z(t)}{\partial S(t-1)}
    \frac{\partial S(t-1)}{\partial Z(t-1)}
    \cdots
    \frac{\partial S(k)}{\partial Z(k)}
    \frac{\partial Z(k)}{\partial W}
\end{align}
in which the multiplication process involves mainly Eq. \eqref{eq-lToZ} and Eq. \eqref{eq-lToS}. As there are element-wise multiplication involved, we cannot simply the computation into a chain of products. After we have computed $\frac{\partial L(t)}{\partial Z(k)}$, we can compute:
\begin{equation}
   \frac{\partial L(t)}{\partial Z(k)} \frac{\partial Z(k)}{\partial W} = S(k-1)^\top\frac{\partial L(t)}{\partial Z(k)}
\end{equation}
Similar to $W$, the gradient with respect to $U$ is
\begin{align}
    \frac{\partial L(t)}{\partial U} 
    &= \sum_{k=1}^t 
    \frac{\partial L(t)}{\partial S(t)}
    \frac{\partial S(t)}{\partial Z(t)}
    \frac{\partial Z(t)}{\partial S(k)}
    \frac{\partial Z(k)}{\partial U}
\end{align}
and after we have computed $\frac{\partial L(t)}{\partial Z(k)}$, we can compute:
\begin{equation}
   \frac{\partial L(t)}{\partial Z(k)} \frac{\partial Z(k)}{\partial U} = X(k)^\top\frac{\partial L(t)}{\partial Z(k)}
\end{equation}
\end{document}
